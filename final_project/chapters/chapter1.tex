\chapter{前言:背景與動機}
\section{背景}
隨著年齡增長,罹患腦中風、缺血性心臟病與癌症的風險也越高,亦為1990年以來WHO統計已開發國家的前三大死因\cite{chu2018},2016年全球的中風病患高達八千萬人,其中近一千四百萬人為新增病例,同年有550萬人死於中風,為全球第二大死因\cite{Johnson2019},其中75\%來自、低收入國家\cite{Collaborators2018},過去十年來(4/6止,衛服部公布至2019,每年國人十大死因)一直列在臺灣國人十大死因的前五名\cite{mhw2018,cdc2020}。

中風、或是腦中風(stroke, cerebrovasular accident),在衛生福利部上,我們又可以看到另一個稱呼「腦血管疾病」。當腦部血管受到阻塞或破裂,腦部細胞欠缺血液的運輸而導致缺氧,細胞進而損傷或死亡,便稱之為腦中風\cite{Johnson2016}。罹患中風後,可能需要面對許多併發症,包含肌肉能力喪失,部分癱瘓、吞嚥困難,中樞神經系統大腦上則有記憶喪失、思覺異常、癲癇等等\cite{Langhorne2000, Kumar2010},其中又有1/3的患者在癒後仍會伴隨這些後遺症\cite{Organization2021}。

除了高死亡率,國家亦需要付出大量的醫療成本,對中風患者進行治療與病後照護,近數十年,腦中風對於全球各國均是不可忽視的問題。

\section{動機}
腦中風已被證實與血壓、糖尿病(肥胖)、年齡密切相關\cite{Boehme2017},而後兩者又可能導致高血壓,因此光探討糖尿病與肥胖,可能就已經足以涵蓋血壓所造成的影響力。故,本次報告除了想要再次證實已知的腦中風與其他病史已知的關係,我們想要瞭解,當不考慮糖尿病史與年齡時,生活環境對血壓與腦中風所造成的關係。
